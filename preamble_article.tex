
% Author       : Yulong Wang
% Date         : 2024-03-12 18:19:11
% LastEditors  : Yulong Wang
% LastEditTime : 2024-06-17 17:21:21
% FilePath     : \Physicsc:\Users\WYL\Documents\template\preamble_article.tex
% Description  : preamble for article tex

% Documentclass
% \documentclass[10pt,a4paper,twocolumn,journal]{IEEEtran}
% \documentclass[12pt,a4paper]{ctexart}

% Chinese environment
\usepackage[UTF8,
scheme=plain,
zihao=-4,
linespread=1.1,
autoindent=true,
heading=true]{ctex}
% \usepackage[utf8]{inputenc}
\DeclareFontFamily{U}{rsfs}{\skewchar\font127 } % refine for error "Font shape `U/rsfs/m/n' in size <15.05624> not available"
\DeclareFontShape{U}{rsfs}{m}{n}{%
   <-6> rsfs5
   <6-8> rsfs7
   <8-> rsfs10
}{}

% Code environment
\usepackage{framed}

% text typography
\usepackage{ulem}       % upgraded underline
% \pagestyle{fancy}
\usepackage{fancyhdr}   % fancy style of heading and foot
\usepackage{indentfirst}% first paragraph indent
\usepackage{color}      % coloring
\usepackage{xcolor}     % more coloring
\usepackage{tipa}       % typography for IPA
\usepackage{ctex}       % pacage for Chinese
\usepackage{xeCJK}
\usepackage{geometry}   % page margins
\geometry{a4paper,left=25mm,right=25mm,top=25mm,bottom=25mm}

\usepackage{titlesec}   % change title's format
\titleformat{\paragraph}{\bfseries}{}{}{\hspace{1em}·}[]

\usepackage{titletoc}   % pacage for contents
% \newfontfamily\sectionef{Timenew-roman}
% \titleformat*{\section}{\large\heiti}
% \titleformat*{\subsection}{\small\heiti}
% \titleformat*{\subsubsection}{\small\kaishu}



% tabular, picture & list packages
\usepackage{enumitem}   % label of list
\usepackage{booktabs}   % Three-line table
\usepackage{array}      % <> embellish
\usepackage{makecell}   % colum break
\usepackage{multirow}   % multi tier
\usepackage{multicol}   % multi colum

\usepackage{graphicx}   % insert pictures
\usepackage{float}      % float enable ⟨placement⟩ = H
\usepackage{stfloats}
% \usepackage{subfig}     % support for subfigure
\usepackage{subfigure}  % another support for subfigure
\usepackage{tikz}       % TikZ painting

\usepackage{enumitem}
\setenumerate[1]{itemsep=0pt,partopsep=0pt,parsep=\parskip,topsep=5pt}
\setitemize[1]{itemsep=0pt,partopsep=0pt,parsep=\parskip,topsep=5pt}

% math environment
\usepackage{amsmath}    % support for math environment from AMS
\usepackage{amsthm}     % optional theorem style and proof
\usepackage{amsbsy}     % bf for part
\usepackage{amssymb}    % symbols form AMS
\usepackage{mathrsfs}   % symbols for mathscr
\usepackage{bm}         % bold symbol
\usepackage{siunitx}	% SI unit

% code environment
\usepackage{listings}
\definecolor{dkgreen}{rgb}{0,0.6,0}
\definecolor{gray}{rgb}{0.5,0.5,0.5}
\definecolor{mauve}{rgb}{0.58,0,0.82}
\lstset{frame=tb,
	language=python,    % 使用的语言
	aboveskip=3mm,
	belowskip=3mm,
	showstringspaces=false,             % 仅在字符串中允许空格
	backgroundcolor=\color{white},      % 选择代码背景,必须加上 color 或 xcolor
	columns=flexible,
	basicstyle = \ttfamily\small,
	numbers=none,       % 给代码添加行号,可取值none, left, right.
	numberstyle=\small \color{gray},    % 行号的字号和颜色
	keywordstyle=\color{blue},
	commentstyle=\color{dkgreen},       % 设置注释格式
	stringstyle=\color{mauve},
	breaklines=true,    % 设置自动断行.
	breakatwhitespace=true,             % 设置是否当且仅当在空白处自动中断.
	escapeinside=``,    %逃逸字符(1左面的键),用于显示中文
	frame=single,       %设置边框格式
	extendedchars=false,                % 解决代码跨页时,章节标题,页眉等汉字不显示的问题
	xleftmargin=2em,xrightmargin=2em, aboveskip=1em,    %设置边距
	tabsize=4           % 将默认tab设置为4个空格
}

% customization
\newcommand{\CC}{\mathbb{C}}
\newcommand{\NN}{\mathbb{N}}
\newcommand{\QQ}{\mathbb{Q}}
\newcommand{\RR}{\mathbb{R}}
\newcommand{\ZZ}{\mathbb{Z}}
\newcommand{\dd}{\mathrm{d}}
\newcommand{\ee}{\mathrm{e}}

\newcommand{\red}[1]{\textcolor{red}{#1}}
\newcommand{\vip}[1]{\textcolor{red}{\textbf{#1}}}

\newenvironment{cnabstract}{\fontsize{9pt}{1}{\heiti 摘\hspace{0.9em}要:}\songti }{}
\newenvironment{key}{\fontsize{9pt}{1}{\heiti 关键词:}\songti }{}
\newenvironment{cnquote}{\begin{quote}\kaishu \small}{\end{quote}}
\newenvironment{prf}{\begin{proof}\small\itshape}{\end{proof}\vspace{0.5em}}
\newenvironment{property}{\noindent \textbf{Property:} \begin{enumerate}}{\end{enumerate}}

\theoremstyle{plain}% default
\newtheorem{xca}{Exercise}[section]

\theoremstyle{definition}
\newtheorem{defn}{Definition}[section]
\newtheorem{thm}{Theorem}[section]
\newtheorem{lem}[thm]{Lemma}
\newtheorem{cor}{Corollary}[section]
\newtheorem{prop}[cor]{proposition}

\theoremstyle{remark}
\newtheorem*{rem}{Remark}
\newtheorem*{sol}{Solution}
\newtheorem*{exmp}{Example}
\newtheorem*{note}{Note}
\newtheorem*{case}{Case}

% Reference & index
% \usepackage{makeidx}      % index for cn
% \makeindex                % cmd: zhmakeindex -s <style>.ist <main>.idx
% \usepackage{imakeidx}     % index for en
% \makeindex[title = Index, columns=2, columnsep = 2em, columnseprule, options= -s en.ist, intoc]   % index customization for en
\usepackage{natbib}             % modern style of citation
\bibliographystyle{plain}    % bibtex refernce style
\setcitestyle{authoryear,open={[},close={]}} %Citation-related commands
\newcommand{ \upcitep}[1]{\textsuperscript{\textsuperscript{\citep{#1} } }} % cite at upper
% \usepackage{bookmark}
\usepackage[
    colorlinks=true,
    linkcolor=blue!70,
    anchorcolor=blue,
    citecolor=blue,
    urlcolor=blue,
]{hyperref}                     % hyperref style